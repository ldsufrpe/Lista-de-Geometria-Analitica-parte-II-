\section{Vetores}
 
%\begin{enumerate}[labelwidth=0.5cm,align=left]

\item Determine as componentes do vetor $\overrightarrow{P_1P_2}$ e esboce-o com seu ponto inicial na origem.
\begin{enumerate}
    \begin{multicols}{2}
    \item $P_1(3,-5)$ e $P_2(0,0)$
    \item $P_1(2,3)$ e $P_2(-1,0)$
    \item $P_1(1,2,3)$ e $P_2(3,5,8)$
    \item $P_1(2,4,3)$ e $P_2(1,-2,2)$
    \end{multicols}
\end{enumerate}
\item Dados $\ub=\lan 1,-1\ran$, $\vb=\lan 2,0\ran$ e $\wb=\lan 3,-2\ran$, determine:% Fabiano 8.4
\begin{enumerate}[leftmargin=*]
    \begin{multicols}{2}
    \item $\|\ub+\vb\|$
    \item $\|\ub - \vb\|$
    \item $\|\ub\|-\|\vb\|$
    \item $\|3\ub\|+5\|\vb\|$
    \item $\|\ub\|+\|-2\vb\|+\|-\wb\|$
    \item $\|\ub\|+2\vb$
    \item $\dfrac{1}{\|\wb\|}\wb$
    \item $\left\|\dfrac{1}{\|\wb\|}\wb\right\|$
    \end{multicols}
\end{enumerate}
  \item[\textcolor{blue}{3-4}] Determine os vetores unitários que satisfaçam as condições dadas.

\item 
    \begin{enumerate}[leftmargin=*]
        \item Mesma direção e sentido que $\vb=-\mathbf{i} + 4\mathbf{j}$.
        \item Sentido oposto a $\vb=6\ib-4\jb+2\kb$.
        \item Mesma direção e sentido que o vetor cujo ponto inicial é $A(-1,0,2)$ e o ponto final é $B(3,1,1)$.%anton 23a
        
       
    \end{enumerate}


\item 
    \begin{enumerate}[leftmargin=*]
        \item Sentido oposto a $\vb=\lan 3,-4\ran$ e a metade do tamanho de $\vb$.
        \item Direção igual e sentido oposto a $\vb=-3\ib +4\jb +\kb$.%anton 24b
    \end{enumerate}
    
 \item Em cada item, determine a forma em componentes do vetor $\vb$ no espaço bidimensional que tenha o comprimento dado e faça o ângulo $\theta$ dado com o eixo $x$ positivo.

    \begin{enumerate}[leftmargin=*]
        \begin{multicols}{2}
        \item $\|\vb\|=3$ ; $\theta=\pi/4$
        \item $\|\vb\|=5$ ; $\theta=120^{\circ}$
        \item $\|\vb\|=2$ ; $\theta=\pi/2$
        \item $\|\vb\|=1$ ; $\theta=180^{\circ}$
        \end{multicols}
    \end{enumerate}
 
  %\begin{enumerate}[leftmargin=*]
      \item Dado que $\|\vb\|=3$, determine todos os valores de $k$ tais que $\|k\vb\|=5$.
      \item Dado que $k =-2$ e $\|kv\|=6$, determine $\| \vb\|$.
      \item Determine o escalar $k$ para que o vetor $\vb=\lan 3k,3k,4k\ran$ seja unitário.
  %\end{enumerate}
  
\item Em cada parte, determine dois vetores unitários no espaço bidimensional que satisfaçam a condição dada.

    \begin{enumerate}[leftmargin=*]
        \begin{multicols}{2}
        \item Paralelo à reta $y = 3x + 2$
        \item Paralela à reta $x + y = 4$.
        \item Perpendicular à reta $y = -5x +1$.
        \end{multicols}
    \end{enumerate}

\item Determine escalares $\alpha_1$, e $\alpha_2$ tais que $\alpha_1\lan 2,-1\ran + \alpha_2\lan -1, -1\ran =\lan 5, -1\ran$.%fabiano 8.7

\item Sejam $\ub =\lan 1,3,-2,1\ran$ e $\vb =\lan 2,0,-1,4\ran$ vetores do $\mathbb{R}^4$. Determine os escalares $\alpha_1$ e $\alpha_2$ que satisfazem as combinações:
    \begin{enumerate}[leftmargin=*]
        \item $\alpha_1\ub + \alpha_2\vb=\lan 8,6,-7,14\ran$
        \item $\alpha_1\ub + \alpha_2\vb=\lan -3,3,0,-7\ran$
    \end{enumerate}
\item Escreva o vetor $\vb=\lan 0,7,6,-3\ran$ como combinação linear dos vetores.
\begin{enumerate}[leftmargin=*]
    \item $\mathbf{a}=\lan 1,2,-1,0\ran$, $\mathbf{b}=\lan 2,3,4,1\ran$ e $\mathbf{c}=\lan 1,3,5,0\ran$

    \item  $\mathbf{e_1}=\lan 1,0,0,0\ran$, $\mathbf{e_2}=\lan 0,1,0,0\ran$ e $\mathbf{e_3}=\lan 0,0,1,0\ran$ e $\mathbf{e_4}=\lan 0,0,0,1\ran$

    \item $\mathbf{a}=\lan 4,0,0,0\ran$, $\mathbf{b}=\lan 0,-1,0,0\ran$ e $\mathbf{c}=\lan 0,0,\frac{1}{2},0\ran$ e $\mathbf{d}=\lan 0,0,0,5\ran$ 
\end{enumerate}

\section{Produtos de Vetores}
\item Em cada item, calcule $\vb\cdot \wb$.
    \begin{enumerate}[leftmargin=*]
        \item $\vb=\lan 1,-2\ran$ e $\wb=\lan -1, 3\ran$
        \item $\vb=\lan 0,1,-2\ran$ e $\wb=\lan 1,-1, 3\ran$
    \end{enumerate}
    
\item Determine os vetores unitários ortogonais ao vetor $\vb=\lan 3,2\ran$.

\item Considerando $\ub=\lan 1,2\ran$, $\vb=\lan 4,-2\ran$ e $\wb=\lan 6,0\ran$. Determine:
    \begin{enumerate}[leftmargin=*]
        \begin{multicols}{2}
        \item $\ub\cdot(7\vb+\wb)$
        \item $\|(\ub\cdot\wb)\wb\|$
        \item $\|\ub\|(\vb\cdot\wb)$
        \item $(\|\ub\|\vb)\cdot\wb$
        \end{multicols}
    \end{enumerate}
\item Determine a projeção ortogonal de $\ub$ na direção de $\vb$.
    \begin{enumerate}[leftmargin=*]
        \item $\ub=\lan 2,1\ran$ e $\vb=\lan -3,2\ran$
        \item  $\ub=\lan 2,6\ran$ e $\vb=\lan -9,3\ran$
    \end{enumerate}
\item Determine $r$ tal que o vetor do ponto $A(1, -1, 3)$ ao ponto
$B(3, 0, 5)$ seja perpendicular ao vetor que parte de $A$ ao ponto $P(r, r, r)$.
\item O que se pode afirmar sobre o ângulo entre os vetores $\ub$ e $\vb$ quando:
    \begin{enumerate}[leftmargin=*]
        \item $\ub\cdot\vb>0$
        \item $\ub\cdot\vb<0$
    \end{enumerate}

\item Em cada parte, determine se $\ub$ e $\vb$ fazem um ângulo agudo, um ângulo obtuso ou se são ortogonais.
    \begin{enumerate}[leftmargin=*]
        \begin{multicols}{2}
        \item $\ub = 7\ib + 3\jb + 5\kb$, $\vb = -8\ib + 4\jb + 2\kb$
        \item $\ub = 6\ib + \jb + 3\kb$, $\vb = 4\ib - 6\kb$
        \item $\ub = \lan1, 1, 1\ran$, $\vb = \lan-1, 0, 0\ran$
        \item $\ub = \lan 4, 1, 6\ran$, $\vb = \lan -3, 0, 2\ran$
        \end{multicols}
    \end{enumerate}

\item Encontre $\ub\cdot\vb$ sabendo que $\|\ub\|=1$, $\|\vb\|=2$, e o ângulo entre $\ub$ e $\vb$ é $\pi/6$.
\item Determine $\ub\cdot(\vb \times \wb)$.
    \begin{enumerate}[leftmargin=*]
        \begin{multicols}{2}
            \item $\ub = 2\ib - 3\jb + \kb$, $\vb = 4\ib + \jb - 3\kb$, $\wb = \jb + 5\kb$
        \end{multicols}
        
    \end{enumerate}
\item Determine o ângulo formado pelas medianas traçadas pelos vértices dos ângulos agudos de um triângulo retângulo isósceles.

\item Determine um vetor simultaneamente ortogonal a $\ub=\lan -1,-1,-1\ran$ e $\vb=\lan 2,2,3\ran$.

\item [\textcolor{blue}{24-25}]Determine a área do paralelogramo de vértices $A$, $B$, $C$ e $D$.
 
\item $A(0,1)$, $B(3,0)$, $C(5,-2)$ e $D(2,-1)$.
\item $A(1,1,0)$, $B(3,1,0)$, $C(1,4,2)$ e $D(3,4,2)$.

\item Determine a área do paralelogramo formados pelos vetores $\ub$ e $\vb$ tais que:
    \begin{enumerate}[leftmargin=*]
        \item $\|\ub\|=3$, $\|\vb\|=4$, e o ângulo entre $\ub$ e $\vb$ é $\pi/3$.
        \item $\|\ub\|=2$, $\|\vb\|=3$ e $\ub\cdot\vb=3\sqrt{2}$
        \item $\|\ub\|=\|\vb\|=1$ e $\ub \perp \vb$
    \end{enumerate}
    

\item [\textcolor{blue}{27-28}]Determine a área do triângulo de vértices $P$, $Q$ e $R$.
    
\item $P(1, 5, -2)$, $Q(0, 0, 0)$ e $R(3, 5, 1)$.

\item $P(2, 0, -3)$, $Q(1, 4, 5)$ e $R(7, 2, 9)$.

 \item Determine a área do paralelogramo $ABCD$ cujas diagonais são $\overrightarrow{AC}=\lan -1,3,-3\ran$ e $\overrightarrow{BD}=\lan -3,-3,1\ran$.
 
 \item [\textcolor{blue}{30-31}] Use um produto misto para determinar o volume do paralelepípedo que tem $\ub$, $\vb$ e $\wb$ como arestas adjacentes.
 
 \item $\ub = \lan 1, -2, 1\ran$, $\vb = \lan 3, 0, -2\ran$, $\wb = \lan 5, -4, 0\ran$
 
 \item $\ub = 5\ib - 2\jb + \kb$, $\vb = 4\ib - \jb + \kb$, $\wb = \ib - \jb$
 
 
\item Considere o paralelepípedo com as arestas adjacentes 
    \begin{align*}
    \ub &=3\ib + 2\jb + \kb\\
    \vb &= \ib + \jb+2\kb\\
    \wb &= \ib + 3\jb + 3\kb
    \end{align*}
\begin{enumerate}[leftmargin=*]
    \item Encontre o volume.
    \item Encontre a área da face determinada por $\ub$ e $\wb$.
    \item  Encontre o ângulo entre $\ub$ e o plano contendo a face determinada por $\vb$ e $\wb$.
\end{enumerate}

 \item[\textcolor{blue}{33-34}] Verifique se $\ub$, $\vb$ e  $\wb$ são coplanares.
  \item $\ub=\lan 2,-1,0\ran $, $\vb=\lan 3,1,2\ran$ e $\wb=\lan 7,-1,2\ran$.
  \item $\ub=3\ib -\jb + 2\kb$, $\vb= \ib +2\jb + \kb$, $\wb=-2\ib + 3\jb + 4\kb$.
  
%\section{Retas e Planos}

\section{Problemas Suplementares I}

\item  Use um teorema da Geometria Plana para mostrar que se $u$ e $v$ forem vetores no espaço bi ou tridimensional, então 
$$\|\ub+\vb\|\leq \|\ub\| +\|\vb\|$$
que é chamada de \textbf{desigualdade triangular para vetores}. Dê alguns exemplos para ilustrar essa desigualdade.

\item Use vetores para provar que o segmento de reta que use o ponto médio de dois lados de um triângulo é paralelo ao terceiro lado e mede a metade do comprimento do terceiro lado.%\anton 63
\item Se $\ub\cdot \mathbf{v}=0$ e $\ub\cdot\mathbf{w}=0$, mostre que $\ub$ é ortogonal a qualquer combinação linear dos vetores $\mathbf{v}$ e $\mathbf{w}$.
\item Mostre que se $\ub+\vb$ é ortogonal a $\ub-\vb$ então $\|\ub\|=\|\vb\|$.
\item Em um triângulo $ABC$, mostre que a altura $h$ relativa ao lado $AB$ é dada por

$$\frac{\|\overrightarrow{AB}\times \overrightarrow{AC}\|}{\|\overrightarrow{AB}\|}.$$
\item Use o Exercício 39 para mostrar que, no espaço tridimensional, a distância $d$ de um ponto $P$ à reta $L$ que passa pelos pontos $A$ e $B$ pode ser expressa como
$$d = \frac{\|\overrightarrow{AP}\times \overrightarrow{AB}\|}{\|\overrightarrow{AB}\|}.$$
\item Sejam $\ub$ e $\vb$ vetores unitários e ortogonais. Mostre que $\|\ub\times \vb\|=1$.
\item É um teorema da Geometria sólida que o volume de um tetraedro é (área da base) x (altura). Use esse resultado para provar
que o volume de um tetraedro cujas arestas são os vetores $\ub$, $\vb$ e $\wb$ é $\frac{1}{6}\|\ub\cdot(\vb\times\wb)\|$.
%\end{enumerate}


